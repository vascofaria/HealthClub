\documentclass[12pt, a4paper]{article}
\include{Config/pacotes}

% Define as dimensoes
\geometry{
	a4paper,
	total={157mm,240mm},
	left=30mm,
	top=20mm,
	right=20mm,
	bottom=40mm
}
\renewcommand{\baselinestretch}{1.5}

\pagestyle{fancy}
\fancyhead[L]{}
\fancyhead[R]{ \includegraphics[width=3cm]{Config/unb.png} }
\renewcommand{\headrulewidth}{0pt}

\begin{document}
%% Informações do candidato
\programa{Pós-Graduação em Geociências Aplicadas e Geodinâmica}
\candidato{Nome do Aluno}
\orientador{Prof. Dr. Nome do Orientador}
\coorientador{Prof. Dr. Nome do Coodientador}
\professordisciplina{Dr. Nome do Professor da Disciplina}
\titulo{Template para a Disciplina Seminário 1 do Instituto de Geociências da Universidade de Brasília}	% Se o título ocupar mais do que duas linhas é preciso modificar o espaçamento no arquivo Config/capa para evitar que a capa ocupe duas páginas
\area{Geoprocessamento e Análise Ambiental}
\linhapesquisa{Avaliação de Dados e Técnicas de Sensoriamento Remoto, Geoprocessamento, Cartografia e Geodésia}
\mes{Agosto}
\ano{2020}

\include{Config/capa}
\onehalfspacing

% ---
% inserir lista de ilustrações
% ---
% \pdfbookmark[0]{\listfigurename}{lof}

\listoffigures
\fancyfoot{}
% \thispagestyle{empty}
\cleardoublepage
% ---
% ---
% ---
% ---
% inserir lista de tabelas
% ---
% \pdfbookmark[0]{\listtablename}{lot}

\vspace{1cm}
\listoftables
\fancyfoot{}

% \thispagestyle{empty}
\cleardoublepage
% ---
% ---
% inserir lista de abreviaturas e siglas
% ---
% \pdfbookmark[0]{\listadesiglasname}{los}
% \begin{siglas}
%   \item[ABNT] Associação Brasileira de Normas Técnicas
%   \item[abnTeX] ABsurdas Normas para TeX
% \end{siglas}

% \pdfbookmark[0]{\contentsname}{toc}
% \thispagestyle{empty}
\tableofcontents
\fancyfoot{}
\cleardoublepage
\newpage

\fancyfoot[R]{
	\footnotesize
	\thepage
}
\fancyfoot[C]{
	\footnotesize
	PROGRAMA DE PÓS-GRADUAÇÃO EM GEOCIÊNCIAS APLICADAS E GEODINÂMICA \\
	Instituto de Geociências - Campus Universitário Darcy Ribeiro \\
	Brasília, DF - CEP 70910-900
}
\renewcommand{\footrulewidth}{1pt}

% Section I
\input{1-intro}    % basic introduction
\input{2-objetivos.tex}
\input{3-fundamentacao_teorica.tex}
\input{4-materiais_e_metodos}
\input{5-resultados}
\clearpage
\input{6-cronograma}
\input{7-viabilidade}

% \input{4-cronograma}
\clearpage
\bibliographystyle{Config/abnt-alf}

\section{REFERÊNCIAS BIBLIOGRÁFICAS}
{ %Disable chapter command
	\renewcommand{\section}[2]{}
	\bibliography{Biblio/Biblio}
}
	
\end{document}